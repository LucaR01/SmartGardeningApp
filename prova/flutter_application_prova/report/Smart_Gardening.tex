% ---------------- RELAZIONE PROGETTO DI PROGRAMMAZIONE AD OGGETTI (OOP) --------
\documentclass[a4paper,12pt]{report}

% ----------------------------- PREAMBLE --------------------------------------- 

\usepackage{lmodern}
\usepackage{alltt, fancyvrb, url}
\usepackage{float}
\usepackage{graphicx}
\usepackage[utf8]{inputenc}
\usepackage{hyperref}
\usepackage{amsmath,amssymb,amsthm}

\usepackage[italian]{babel}

\usepackage[italian]{cleveref}

\usepackage{comment}
\usepackage{microtype}
\usepackage{fancyhdr}

\usepackage[scaled=.92]{helvet}
\usepackage[T1]{fontenc}

\usepackage{lscape}

\usepackage{subcaption}

% hyperref settings
\hypersetup{
	colorlinks=true,
	linkcolor=black, %blue
	filecolor=magenta,      
	urlcolor=cyan,
	pdftitle={Sharelatex Example},
	bookmarks=true,
	pdfpagemode=FullScreen,
}

\usepackage{titlesec}
%\usepackage{titletoc}

\setcounter{tocdepth}{4}
\setcounter{secnumdepth}{1}

\titleformat{\section}{\normalfont\Large\bfseries\centering}{}{0pt}{} % Questo serve per togliere il numero dalla section

% ----------------------------- PREAMBLE END -----------------------------------

\makeindex

\title{\textbf{Tirocinio} \\[1.5ex] Smart Gardening App}
\author{Luca Rengo}

\begin{document}
	
	\makeatletter
	\begin{titlepage}
		\begin{center}
			%\includegraphics[width=0.7\linewidth]{images/logo/}\\[4ex] %TODO: mettere logo app
			{\Huge  \@title }\\[3ex] 
			{\large  \@author}\\[3ex] 
			{\large Maggio | Giugno | Luglio | Agosto | Settembre 2022}
		\end{center}
	\end{titlepage}
	\makeatother
	\thispagestyle{empty}
	\newpage
	
	%\maketitle
	
	\tableofcontents
	
	\newpage
	
	% \input: import the commands from filename.tex to target file.
	
	% \include: does a \clearpage and does an \input.
	
% ============================== INTRODUZIONE =========================================
	
	\section{Introduzione}
	
	\begin{comment}
	\begin{figure}[ht] 
		\centering
		\includegraphics[width=1.2\textwidth, height=1.2\textheight, keepaspectratio]{./images/}
		\caption{Homepage}
		\label{fig:homepage}
	\end{figure}
	\end{comment}

% ============================== HOMEPAGE =============================================
	
% ============================== GUIDA UTENTE =========================================
	
	\newpage
	
	\section{Guida utente}
	
	\textsf{\small Qui, di seguito verranno indicate le procedure e le istruzioni da eseguire per poter avviare correttamente l'applicazione: } \\
	
	\begin{itemize}
		\item \textsf{\small }
	\end{itemize}
\end{document}

% ====================================================================================


